%%%%%%%%%%%%%%%%%%%%%%%%%%%%%%%%%%%%%%%%%
% AGU 2025 Landscape Poster Template
% LaTeX Template
%
% This template is adapted for the American Geophysical Union (AGU) 2025 Fall Meeting.
%
% Original a0poster class created by:
% Gerlinde Kettl and Matthias Weiser (tex@kettl.de)
%
% Modified and Adapted for AGU 2025 by:
% Jivesh Dixit (jiveshdixit@gmail.com)
%
% License:
% CC BY-NC-SA 3.0 (http://creativecommons.org/licenses/by-nc-sa/3.0/)
%
%%%%%%%%%%%%%%%%%%%%%%%%%%%%%%%%%%%%%%%%%

\documentclass[a0,landscape]{a0poster}
% control margins here
\usepackage{geometry}
 \geometry{
 a0paper,
 landscape,
 left=5cm,
 top=4.5cm,
 right=5cm,
 bottom=0cm
 }
\addtolength{\textwidth}{2.5cm} % width of text can be adjusted if margins above are adjusted
\usepackage[caption=false,font=normalsize,labelfont=sf,textfont=sf]{subfig}
\usepackage{textcomp}
\usepackage{stfloats}
\usepackage{url}
\usepackage{verbatim}
\usepackage{graphicx}
\usepackage{booktabs} 
\usepackage[table]{xcolor}
\usepackage{colortbl}
\usepackage{orcidlink}
\usepackage{hyperref}
\usepackage{fontawesome5}
\usepackage{multicol} % This is so we can have multiple columns of text side-by-side
\columnsep=3em % This is the amount of white space between the columns in the poster
\columnseprule=0pt % This is the thickness of the black line between the columns in the poster

% UZH colours (Preserved from original theme)
\usepackage[svgnames]{xcolor}
\definecolor{uzhblau100}{RGB}{0, 40, 165}
\definecolor{uzhblau80}{RGB}{51,83,183}
\definecolor{uzhockerrot100}{RGB}{220, 96, 39}
\definecolor{uzhockerrot80}{RGB}{227, 128, 82}
\definecolor{uzhflaschengruen100}{RGB}{42, 127, 98}
\definecolor{uzhflaschengruen80}{RGB}{86, 157, 133}
\definecolor{conclusion}{RGB}{204,212,237} % the conclusion box colour

\usepackage{ifthen} % needed to stop horizontal line above 'Conclusions' section
\usepackage{graphicx} % Required for including images
\graphicspath{{figures/}} % Location of the graphics files
\usepackage{mwe,tikz}\usepackage[percent]{overpic} % overlay your photo over the background
\usepackage{booktabs} % Top and bottom rules for table
\usepackage[font=small,labelfont=bf]{caption} % Required for specifying captions to tables and figures
\usepackage{amsfonts, amsmath, amsthm, amssymb} % For math fonts, symbols and environments
\usepackage{wrapfig} % Allows wrapping text around tables and figures
\usepackage{stfloats}
\usepackage{fontspec} % custom fonts
\usepackage[skins,breakable]{tcolorbox}

% Font Configuration
\defaultfontfeatures[Palatino]
{
    Extension = .ttf,
    UprightFont = font/LT_41167,
    BoldFont = font/LT_41169,
    ItalicFont  = font/LT_41168,
    BoldItalicFont = font/LT_41170,
}
\defaultfontfeatures[TheSans]
{
    Extension = .otf,
    UprightFont = font/TheSans-LP5Plain,
    BoldFont = font/TheSans-LP7Bld,
    ItalicFont  = font/TheSans-LP5PlainIT,
    BoldItalicFont = font/TheSans-LP7BldIT,
}
\setmainfont{TheSans} % choose your font here

\usepackage[onehalfspacing]{setspace} % remove for single spacing
\usepackage{tcolorbox} % for the conclusions box
\usepackage{blindtext} % for dummy text
\usepackage[export]{adjustbox} % allow floating a graphic right
\usepackage{titlesec} % customising section titles
\usepackage{nameref} % package and command to get the name of the current section (for ifthen)
\usepackage{needspace} % for \needspace command
\usepackage{eso-pic}

% Fix for A0 poster size in PDF (Essential for modern TeX engines)
\pdfpagewidth=\paperwidth
\pdfpageheight=\paperheight

\makeatletter
\setlength{\pdfpagewidth}{\paperwidth}
\setlength{\pdfpageheight}{\paperheight}
\makeatother

\begin{document}

% define how our section titles will look (with ruled line)
\titleformat{\section}
  {\needspace{0.6\baselineskip}\sectionrule\huge\bfseries}
  {\color{uzhblau100}\thesection.}
  {1em}
  {\color{uzhblau100}}

% draw horizontal line before section unless it is conclusions
\makeatletter
\newcommand{\sectionrule}{%
 \ifthenelse{\equal{\@currentlabelname}{Conclusions}}
  {}
  {\vspace*{-\baselineskip}
   \vrule height 1pt depth 1pt width \linewidth\vskip0.4pt
   \bigskip}%
}
\makeatother


%----------------------------------------------------------------------------------------
%	POSTER HEADER 
%----------------------------------------------------------------------------------------
\noindent % Prevents indentation

% --- BLOCK 1: Title & Authors (Reduced to 55% width to make room) ---
\begin{minipage}[c]{0.55\linewidth}
    \raggedright
    % TITLE
    {\fontsize{80pt}{95pt}\selectfont \color{uzhblau100}\textbf{AGU 2025 Poster Template: Title goes here}\par}
    
    \vspace{0.8cm} 
    
    % AUTHORS
    {\huge \underline {\textbf{Jivesh Dixit\textsuperscript{1}}}, and Co-Author Name\textsuperscript{1} \hspace{1cm} (\LARGE \textsuperscript{1}CAS, IIT Delhi)}
\end{minipage}%
\hfill 
% --- BLOCK 2: Personal Photo (New Column - 15% width) ---
\begin{minipage}[c]{0.15\linewidth}
    \centering
    % OPTION A: Square/Rectangular Image
    % \includegraphics[width=0.8\linewidth, valign=c]{example-image}
    
    % OPTION B: Circular Crop (Professional Look) - Requires TikZ
    \begin{tikzpicture}
        \node[circle, draw=uzhblau100, line width=2pt, inner sep=0pt, minimum size=8cm, path picture={
            \node at (path picture bounding box.center) {
                \includegraphics[width=8cm]{figures/Silhouette.png} % <--- PUT YOUR PHOTO FILENAME HERE
            };
        }] {};
    \end{tikzpicture}
\end{minipage}%
\hfill
% --- BLOCK 3: Logos (Reduced to 25% width) ---
\begin{minipage}[c]{0.25\linewidth}
    \raggedleft 
    % Image 1
    \includegraphics[width=0.45\linewidth, valign=c]{figures/agu25.jpg} 
    \hspace{0.5cm} 
    % Image 2
    \includegraphics[width=0.50\linewidth, valign=c]{figures/IITD_logo.png}
\end{minipage}

\vspace{0cm} % Space before columns start


%----------------------------------------------------------------------------------------
%	POSTER BODY
%----------------------------------------------------------------------------------------

\begin{multicols}{4} % 4 Columns

%----------------------------------------------------------------------------------------
%	INTRODUCTION
%----------------------------------------------------------------------------------------

\section{Introduction}

\large
\blindtext[1] % Generates dummy Latin text

This is an example citation \cite{Eyring2016Overview}. The project aims to improve existing methodologies by applying advanced algorithms \cite{Boer2016Decadal}.

\section{Data and Methodology}
\subsection {Data Acquisition}
\blindtext[1]

\subsection{Methodology Framework}
The proposed architecture (Figure \ref{fig:method_diag}(a)) is illustrated in comparison to the baseline in Figure \ref{fig:method_diag}(b).

\begin{center}\vspace{1cm}
    % Placeholder image
    \includegraphics[width=1.00\linewidth]{figures/licensed-image.jpg}
    \captionof{figure}{(a) Diagram of the proposed architecture and (b) Flowchart of the algorithm implementation.}
    \label{fig:method_diag} 
\end{center}\vspace{1cm}

We evaluate the model using standard metrics such as Root Mean Square Error (RMSE) and correlation coefficients.

\begin{equation}
\text{Correlation} = \frac{\sum (x - \bar{x})(y - \bar{y})}{\sqrt{\sum (x - \bar{x})^2 \sum (y - \bar{y})^2}}
\label{eq:corr}
\end{equation}

\begin{equation}
\text{RMSE} = \sqrt{\frac{1}{N} \sum_{i=1}^{N} (P_i - O_i)^2}
\label{eq:rmse}
\end{equation}

\section{Results}

The results indicate a significant improvement over the baseline models.

\begin{center}\vspace{1cm}
    \includegraphics[width=1.00\linewidth]{figures/Fig1.jpg}
    \captionof{figure}{Comparison of model performance metrics across different scenarios.}
\end{center}\vspace{1cm}

\blindtext[1]

\begin{center}\vspace{1cm}
    \includegraphics[width=1.00\linewidth]{figures/Fig2.jpg}
    \captionof{figure}{Visual representation of the output data distribution.}
\end{center}\vspace{1cm}

%----------------------------------------------------------------------------------------
%	CONCLUSIONS BOX
%----------------------------------------------------------------------------------------

\vspace{2cm}
\begin{tcolorbox}[width=1\linewidth,colback={conclusion},frame empty,boxsep=1cm, breakable]
\section{Conclusions}
\begin{itemize}
    \item {The proposed method significantly outperforms traditional approaches in terms of accuracy.}
    \item {Data analysis reveals critical insights into the underlying patterns of the system.}
    \item {Future work will focus on optimizing the algorithm for real-time applications.}
    \item {Integration with other datasets could provide further validation of these findings.}
\end{itemize}
\end{tcolorbox}    


%----------------------------------------------------------------------------------------
%	REFERENCES
%----------------------------------------------------------------------------------------
\singlespacing
\small
% Note: You need a .bib file named 'sample.bib' for this to work
\nocite{*} 
\bibliographystyle{plain} 
\bibliography{sample} 

%----------------------------------------------------------------------------------------

\end{multicols}

%----------------------------------------------------------------------------------------
%	FOOTER
%----------------------------------------------------------------------------------------
\AddToShipoutPictureFG*{%
    \AtPageLowerLeft{%
        \begin{tikzpicture}[overlay,remember picture]
            \node[
                anchor=south,
                fill=uzhblau100, 
                text=white, 
                text width=\paperwidth, 
                minimum width=1.05\paperwidth, % Wider than page to avoid gaps
                minimum height=0.3cm,  
                inner sep=0pt, 
                outer sep=0pt,
                yshift=-4cm % Adjust this to move footer up/down
            ] at (0.5\paperwidth, 0) {  
                \vspace*{1.75cm}
                \begin{minipage}{0.95\paperwidth}
                    \centering
                    \LARGE 
                    \color{white}
                    
                    % Left: Author Website
                    \begin{minipage}{0.30\linewidth}
                        \flushleft
                        \hspace {4cm}
                        \href{https://www.jiveshdixit.in/}{\textbf{Jivesh Dixit (IIT Delhi)}\hspace{0.1cm}\faExternalLink*}
                    \end{minipage}%
                    \hfill
                    % Center: Paper ID (Example AGU ID)
                    \begin{minipage}{0.30\linewidth}
                        \centering
                        % Replace with your actual AGU Link and ID
                        \href{https://agu.confex.com/agu/agu25/meetingapp.cgi/Paper/1957414}{%
                            \textbf{AGU25-A51O-XXXX} \hspace{0.1cm} \faExternalLink*
                        }
                    \end{minipage}
                    \hfill
                    % Right: Email
                    \begin{minipage}{0.30\linewidth}
                        \flushright
                        \href{mailto:jiveshdixit@gmail.com}{%
                            \faEnvelope \hspace{0.2cm} jiveshdixit@gmail.com%
                        }
                    \end{minipage}
                \end{minipage}
            };
        \end{tikzpicture}%
    }%
}

\end{document}